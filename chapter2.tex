\chapter{Literature Survey}

The paper \cite{ran2023} presents a new method for fatigued driving using a deep learning model that combines the lightweight nature of YOLOv5s with 3D facial key feature extraction. This study shows that the improved YOLOv5 architecture, equipped with ShuffleNetV2\_BD and maxpool cross-scale feature aggregation (M-CFAM), improves the detection accuracy while reducing the computational cost. Vital signs such as Early Response Rate (EAR) and Marginal Response Rate (MAR) are evaluated for important fatigue indicators such as eyes closed and yawning.
The paper notes that PERCLOS (percentage of eyes closed) is useful in detecting fatigue in the short term. However, the long-term monitoring or intervention effects of different factors on the working model are also discussed.
In addition, this model primarily focuses on immediate detection and does not consider the influence of different lighting conditions or environmental factors that may affect the accuracy in real-world use. The focus is on additional measures and other factors that may affect the reliability of driving fatigue intensification.

The paper \cite{guo_markoni} proposes a hybrid model combining MTCNN for spatial feature extraction and LSTM and TSC-LSTM for temporal sequence analysis (drowsiness detection), enhancing detection accuracy with less noise. The cost computation is compared with original VGG11 architecture to find the fastest and most accurate CNN.

A systematic literature review was conducted to identify studies on eye-activity-based driver drowsiness detection (DDD) systems \cite{kolus2024}. Relevant studies were selected from four databases, focusing on drowsiness, eye activity, and performance measures. Key data on ocular parameters, technologies, and performance metrics were extracted and analyzed, with a classification of eye activity measures. Future directions for DDD systems were also discussed.

The study \cite{madni2024} proposed a VGLG approach that combines VGG-16 for feature extraction and LGBM for transfer feature generation. Four machine learning models and two deep neural networks were validated using k-fold cross-validation and hyperparameter tuning. The approach utilized a standard dataset comprising 4,103 eye images of drivers, meticulously labeled as either open or closed eye movements.

The paper \cite{ramzan2024} suggested a combination of machine learning and deep learning models using HOG features, PCA, and a custom 30-layer CNN architecture. YawDD Video Dataset, with videos of drivers showing normal, talking, and yawning behaviors.

The paper \cite{venkatsewarulu_reddy} proposed a shallow CNN architecture using limited training data for drowsiness detection based on visual cues like eyelid closure.

Head Pose Estimation Through Euler Angles and Deep Learning Ruiz et al. \cite{ruiz2018} introduce a convolutional neural network (CNN) model that predicts head pose angles—specifically pitch, roll, and yaw—directly from image intensities. This approach bypasses traditional keypoint-based methods, which rely on aligning 2D-3D face models and can be error-prone when landmarks are obscured or inaccurately detected. In Ruiz’s study, the CNN model’s multi-loss function simultaneously classifies and regresses each head pose angle, a method shown to significantly improve accuracy. Using a large, synthetically expanded dataset for training, the model achieves state-of-the-art results on benchmarks such as the AFLW2000 and BIWI datasets.
However, certain limitations are acknowledged, such as a reduction in performance under extreme head poses and on low-resolution images, which points to an ongoing need for robustness improvements, especially for applications like surveillance and driver monitoring.
GRU-Based Temporal Prediction Model for Stock Indices The study by \cite{ruiz2018} proposes StockNet, a GRU-based (Gated Recurrent Unit) model designed to predict stock indices by integrating injection and investigation modules that reduce overfitting through data augmentation.
StockNet utilizes temporal dependencies within financial time series data, showing that GRU architectures can effectively capture sequential relationships. Despite achieving strong predictive accuracy, StockNet’s reliance on specific data distributions and the need for further generalization to diverse market conditions remain areas for development. The study illustrates the GRU model's capacity to handle complex, time-sensitive financial forecasting tasks, while also suggesting that continuous model adaptation is essential for dynamic financial environments.

The study in \cite{ye2023} described an empirical relation between different Euler angles --- pitch, roll and yaw and extreme deflection of driver's head. As per the study, when the yaw angle is greater than 20°, the pitch angle is greater than 21°, and the roll angle is greater than 20.5°, the driver’s head position could be concluded to be deviated excessively.
We can use this in empirical relation after obtaining the Euler's angles of the head from \cite{ruiz2018}, and detect driver's drowsiness using head position.

